Bien qu\textquotesingle{}utilisé par un nombre considérable de passionnés et professionnels de l\textquotesingle{}électronique dans le monde, l\textquotesingle{}oscilloscope est un outil qu\textquotesingle{}on a tendance à utiliser sans forcément en comprendre le fonctionnement. Mais que se passe t-\/il réellement à l\textquotesingle{}intérieur de ces machines ? C\textquotesingle{}est la question qui nous a poussés à nous diriger vers la réalisation d\textquotesingle{}un oscilloscope numérique au cours de ce mini-\/projet microprocesseurs, l\textquotesingle{}objectif fixé étant de réaliser un appareil permettant de relever différents types de signaux et d\textquotesingle{}en afficher le résultat sur un écran L\+C\+D. 